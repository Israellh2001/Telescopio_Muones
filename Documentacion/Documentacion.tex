\documentclass[a4paper,10pt]{article}
\usepackage[utf8]{inputenc}

\author{Israel L. Heynes\thanks{israellh2001@icloud.com}}
\title{Telescopio de Muones Cosmicos}

\begin{document}
    \maketitle
    \tableofcontents

    \section{Introducción}

	Este programa fue realizado para estudiar el comportamiento de los muones generados por rayos cosmicos cuando atraviesan un material en concreto. Para ello se creo una serie de \emph{Drift Chambers} con las cuales poder deducir las trayectorias de los muones y analizar los cambios al atravezar el material.

	El simulador utiliza la serie de librerias en \verb|C++| desarrolladas en \emph{CERN}, originalmente para \verb|Fortran|, conocida como \verb|Geant4|.

	Para este programa se recomienda utilizar una distribución de \verb|GNU/Linux| con los siguientes programas instalados:

	\begin{itemize}
		\item \verb|Geant4| version 11.0.0 o mayor
		\item Algun compilador para \verb|C++| preferentemente \verb|g++| version 12.0.0 o mayor.
		\item La interfaz \verb|cmake| version 3.0.0 o mayor.
	\end{itemize}

    \section{Comó compilar y correr el programa}
    	Una vez cumplidos los requisiston ya expuestos se debera estar seguro de tener las librerías de \verb|Geant4| indicadas en el \verb|PATH| para quen la compilación se complete con exito. Los pasos para compilar el programa son los siguientes:

	\begin{enumerate}
		\item Crear un directorio llamado \verb|build| en la carpeta raiz del programa.
		\item Correr el siguiente comando \verb|$ cmake ..|, el programa no requiere ninguna otra configuración extra.
		\item Correr el siguiente comando \verb|$ make| esto comenzara la compilación del programa y arrojara un ejecutable llamado \verb|telescopio| que sera el programa.
	\end{enumerate}

    \section{Clases}
        Las simulaciónes en \verb|Geant4| utilizan un "esqueleto" que el kernel utiliza para hacer las simulaciones. Sin embargo es importante señalar que ninguna parte de este "esqueleto" es absolutamente necesaria.
	Para realizar el "esqueleto" de este programa se utilizan seis clases que el kernel utiliza para:
	
	\begin{itemize}
		\item Crear la geometría de la simulación, esto incluye generar los materiales de los cuales estan hechos los objetos.
		\item Describir la pistola de particulas, esto incluye que particulas se disparan, con que energía y en que dirección.
		\item Saber que hacer con los datos que se obtienen de un evento (colisión).
	\end{itemize}

	Las clases estan relacionadas entre ellas y para que una funcione correctamente debe haber metodos "obligatorios" que se heredan de las clases en \verb|Geant4|.

	Al kernel de \verb|Geant4| se inicializa en la clase principal \verb|main.cc| y aquí se indican las estrucutras de la lista de fisica, la geometria y el generador de eventos.

	Las clases en el programa se conforman de dos archivos \verb|.hh| y \verb|.cc| esto se hereda de la estructura de las clases en \verb|C++|. En resumen, en los archivos \verb|.hh| se declara la estructura general de las clases así como algunos atributos y los metodos de la clase; en los archivos \verb|.cc| se escribe que hacen los metodos.
	
	\subsection{\verb|DetectorConstruction|}

		Esta clase es la encargada de definir la geometría del sistema. Es la primera en cargar \verb|Geant| y es la única con la que el programa no correría. Así que si esta no esta estructurada correctamente el programa no funcionara.
		
		Dentro de la clase hay dos metodos escenciales (ademas de los metodos de creación y aniquilación) estos son \verb|Construct()| y \verb|ConstructSDandField()|. El primer metodo es el encargado de construir la geometríca, definir los materiales, etc. El segundo metodo es el encargado de definir los objetos encargados de detectar cuando una particula pasa atravez de él.

		Para definir un objeto como sensible en la geometría tenemos que definir una clase que se encarga de generar un molde para los hits esa es la clase es la siguiente.

		\subsubsection{\verb|SensitiveDetector|}

			En esta clase se utiliza el \verb|touchable| un objeto que genera el kernel de\verb|geant| y nos da acceso a los datos que se obtienen en un hit. Esta clase utiliza esos datos para general un molde para los hits que se obtendran de los objetos sensibles que se declararon.

	\subsection{\verb|PhysicsList|}

		En esta clase se declaran las listas de physicas, esto es, los fenomenos que queremos medir en nuestra simulación.

	\subsection{\verb|ActionInitializer|}
		
		Esta clase es la encargada de definir que ocurre cuando se llama a un evento en la simulación, esto incluye la pistola de particulas y como procesar los hits que estos generan al chocar con objetos que sean sensibles, esto es que esten definidos como sensores en la geometría.

		\subsubsection{\verb|PrimaryGeneratorAction|}
			
			Esta clase es la encargada de definir la pistola de particulas, aquí se definen los datos de las particulas que se disparan en cada evento como puede ser: el momento, la energía, la dirección y el punto de partida.

		\subsubsection{\verb|RunAction|}

			Esta clase define como se procesaran los hits obtenidos. Aquí es en donde podemos obtener los datos y exportarlos a un archivo binario, csv, root, excel, etc.

    \appendix
        \section{Instalación de Geant4}

\end{document}
